\documentclass[a4paper, 11pt]{report}
\usepackage[utf8]{inputenc}
\usepackage{units}    % units
\usepackage{nicefrac} % fractions within text
\usepackage[german]{babel}
\usepackage{tikz}
\usetikzlibrary{calc, intersections, decorations.pathreplacing, calligraphy, patterns, arrows.meta, positioning, shapes}
\usepackage{pgf, pgfplots}
\pgfplotsset{compat = 1.18} 
\usepackage{algorithm, algorithmicx, algpseudocode} % Pseudocode
\usepackage{amsmath, amssymb, amsthm} % definitions
\usepackage{bbm}
\usepackage{pict2e} % angle symbol
\usepackage[pdfpagelabels,colorlinks,allcolors=blue]{hyperref} % Hyperlink referencing
\usepackage{subcaption} % subfigure captions
\usepackage{bm} % Bold math letters
\usepackage{tipa, stmaryrd} % eta symbol
\usepackage{multirow} % Multirow table
\usepackage{booktabs} % Better tables
\usepackage{cancel}

\algrenewcommand\algorithmicrequire{\textbf{Input:}}
\algrenewcommand\algorithmicensure{\textbf{Output:}}
\algnewcommand\AND{\textbf{and }}
\algnewcommand\OR{\textbf{or }}
\algnewcommand\NOT{\textbf{not }}

%%%%%%%%%%%%%%%%%%%%%%%%%%%%%%%%%%%%%%%%%%%%%%%%%%%%%%%%%%%%%%%%%%%%%%%%%%%%%%%%%%%%

\renewcommand{\geq}{\geqslant}
\renewcommand{\leq}{\leqslant}
\renewcommand{\phi}{\varphi}
\renewcommand{\epsilon}{\varepsilon}
\renewcommand{\theta}{\vartheta}
\renewcommand{\eta}{%
    \raisebox{-0.7mm}[0.8\height][\width]{
        \parbox{0.5mm}{
            \makebox[0.1mm]{
            \begin{tikzpicture}%
                \draw node at (-0.04, 0) {\scalebox{0.9}{\textlhtlongi}}; 
                \draw node at (0.09, -0.08) {\rotatebox[origin=c]{15}{\scalebox{1}[-1]{\(\Rbag\)}}};
            \end{tikzpicture}%
            }
        }
    }\hspace*{1pt}
}


\DeclareSymbolFont{myletters}{OML}{ztmcm}{m}{it}
\DeclareMathSymbol{\uplambda}{\mathord}{myletters}{"15}
\renewcommand{\lambda}{\uplambda}
\newcommand{\dx}{\textup{d}}
\newcommand{\abs}[1]{\left|#1\right|}
\newcommand{\norm}[1]{\abs{\abs{#1}}}
\def\i{\boldsymbol{i}}
\newcommand{\commend}[1]{}
\newcommand{\angl}{
    \raisebox{-0.5mm}{
        \begin{tikzpicture}[line width = 0.2]
            \draw [rounded corners = 0.1] (27:0.35) -- (0, 0) -- (-25.5:0.35);
            \draw (-25.5:0.2) arc (-27:25.5:0.2);
        \end{tikzpicture}
    }\hspace*{-1.5mm}
}

\let\cleardoublepage\clearpage
\DeclareRobustCommand{\lltriangle}{%
  \begingroup
  \setlength{\unitlength}{1ex}%
  \begin{picture}(1,1)
  \polygon(0, 0)(1.3, 0)(0, 1.3)
  \end{picture}%
  \endgroup
}

\newtheoremstyle{break}{}{}{\itshape}{}{\bfseries}{.}{\newline}{}
\newtheoremstyle{rem}{}{}{}{}{\bfseries}{.}{\newline}{}
  
\theoremstyle{break}
    \newtheorem{definition}{Definition}
    \newtheorem{theorem}{Satz}
    \newtheorem{lemma}{Lemma}
    \newtheorem{corollary}{Korollar}

\theoremstyle{rem}
    \newtheorem{example}{Beispiel}
    \newtheorem{remark}{Bemerkung}
    \newtheorem{proposition}{Proposition}

\renewcommand\qedsymbol{q.e.d.}
\let\oldref\ref
\AtBeginDocument{
    \renewcommand{\ref}[1]{[\oldref{#1}]}
    \renewcommand{\eqref}[1]{(\oldref{#1})}
}

\begin{document}
    \begin{titlepage}
        \begin{center}
            \LARGE{\textsc{}}
            
            \vfill
            
            \LARGE{\emph{Studienprojekt}}
            
            \vspace{8mm}
            
            \huge{\textbf{Der H-Kobordismus-Satz}}
            
            \vspace{8mm}
            
            \LARGE{Torge Graner}
            
            \vspace{32mm}
            
            \large{\today}
            \vfill
            
            \begin{tabular}{ll}
              \large
              Betreuer: & \large Prof. Dr. Oliver R\"ondigs
            \end{tabular}
        \end{center} 
    \end{titlepage}
    \addtocontents{toc}{\protect\thispagestyle{empty}}
    \tableofcontents
    \thispagestyle{empty}

    \setcounter{chapter}{-1}
    \setcounter{page}{0}
    \newpage % -- Ausbessern --
    \chapter{Einleitung}\label{chp:introduction}
        Der H-Kobordismus-Satz ist ein wichtiges Resultat der Differentialtopologie und erm\"oglicht unter anderem, in hinreichend gro\ss er Dimension zu entscheiden, ob zwei Mannigfaltigkeiten zueinander diffeomorph sind. Mit diesem l\"asst sich unter anderem die verallgemeinerte Poincar\'e-Vermutung f\"ur Dimensionen \(n\geq5\) beweisen, wof\"ur Stephen Smale 1966 die Fields-Medaille erhielt. Weitere Beweise des Satzes lassen sich zum Beispiel in den B\"uchern \glqq Differential Manifolds\grqq{} von Antoni Kosinski \cite{kosinski2013differential}, \glqq Lectures on the h-Cobordism Theorem\grqq{}, welches auf den Niederschriften von L. Siebenmann und J. Sondow eines Seminares von John Milnor basiert \cite{milnor1965hcobordism}, oder \glqq A Basic Introduction to Surgery Theory\grqq{} von Wolfgang L\"uck \cite{lück2002surgery} finden. Der Ansatz von Milnor ist hierbei einer sehr technischen Natur kommt komplett ohne die Erw\"ahnung von Henkelzerlegungen aus. Ebenso vollst\"andig und technisch ist der Ansatz von Kosinski. Da der H-Kobordismus Satz in L\"ucks Buch nur einen kurzen Teil einnimmt, ist der dort dargelegte Beweis etwas k\"urzer und nicht auf Details fokussiert. Ziel der folgenden Arbeit sei, die Beweiskette von L\"uck zu verfolgen, wobei einige Details und Definitionen eher an Kosinski angelehnt seien. Grundlagen \"uber glatte Mannigfaltigkeiten lassen sich zum Beispiel in \glqq Einf\"uhrung in die Differentialtopologie\grqq{} von Klaus J\"anich und Theodor Br\"ocker \cite{bröcker1990difftop}, \glqq Differential Topology\grqq{} von Morris Hirsch \cite{hirsch2012difftop} oder \glqq Introduction to Smooth Manifolds\grqq{} von John M. Lee \cite{lee2013introduction}. Jegliche Ergebnisse der (algebraischen) Topologie k\"onnen in \glqq Algebraic Topology\grqq{} von Allen Hatcher \cite{hatcher2002algebraic} gefunden werden. Letztlich sei noch angemerkt, dass der Inhalt des Appendix zwar f\"ur die eigentliche Arbeit ignoriert werden kann, jedoch einen enormen Zeitaufwand erfordert hat.
    
    \newpage % -- Ausbessern --
    \chapter{Mannigfaltigkeiten}\label{chp:manifolds}
        Als glatte Mannigfaltigkeit sei im Weiteren stets ein zweitabz\"ahlbarer Haus\-dorff-Raum bezeichnet, der sich lokal in den \({\mathbb{K}\in\{\mathbb{R}^{n+1},\mathbb{R}^n\times\mathbb{R}_{\geq0},\mathbb{R}_{\geq0}^{n+1}\}}\) einbetten lasse und mit einer glatten Struktur versehen sei. Im ersten Fall handelt es sich um eine gew\"ohnliche glatte Mannigfaltigkeit, im zweiten um eine glatte Mannigfaltigkeit mit Rand und im dritten um eine glatte Mannigfaltigkeit mit Ecken. Insbesondere sind Mannigfaltigkeiten mit Ecken nicht unbedingt Mannigfaltigkeiten mit Rand. \textbf{Alle Mannigfaltigkeiten seien im Folgenden glatte, kompakte Mannigfaltigkeiten mit (m\"oglicherweise leerem) Rand}. Mannigfaltigkeiten mit leerem Rand hei\ss en geschlossen. 

\section{Kobordismen}\label{sec:cobordism}
    \begin{definition}[Kobordismus]\label{def:cobordism}
        Seien \({\mathcal{M}^n}\) und \({\mathcal{N}^n}\) geschlossene Mannigfaltigkeiten, so hei\ss e eine Mannigfaltigkeit \({\mathcal{W}^{n+1}}\) mit einer Zerlegung \({\partial\mathcal{W}\cong\mathcal{M}\sqcup\mathcal{N}}\) ein Kobordismus von \({\mathcal{M}}\) zu \({\mathcal{N}}\).
    \end{definition}
    Es werde gelegentlich auch die Notation \({\partial_-\mathcal{W}:=\mathcal{M}}\) und \({\partial_+\mathcal{W}:=\mathcal{N}}\) genutzt.

    \begin{definition}[Diffeomorphie zweier Kobordismen]\label{def:diffeomorphic_cobordism}
        Zwei Kobordismen \({\mathfrak{W}=\left(\mathcal{W},\mathcal{M},\mathcal{N}\right)}\) und \({\mathfrak{W}^{\prime}=\left(\mathcal{W}^{\prime},\mathcal{M},\mathcal{N}^{\prime}\right)}\) hei\ss en diffeomorph relativ zu \({\mathcal{M}}\) (im Folgenden einfach diffeomorph), falls das Diagramm 
        \begin{center}
            \begin{tikzpicture}[scale = 1.2]
                \draw 
                    node (A) at (-1.5, -1) {\({\mathcal{N}}\)}
                    node (B) at (-0.75, 0) {\({\mathcal{W}}\)}
                    node (C) at (0.75, 0) {\({\mathcal{W}^{\prime}}\)}
                    node (D) at (1.5, -1) {\({\mathcal{N}^{\prime}}\)}
                    node (E) at (0, -1) {\({\hspace{-4pt}\mathcal{M}}\)}
                    (B) -- node [above] {\({\sim}\)} node [below] {\({\exists}\)} (C);
                \path
                    (A) -- node [sloped] {\({\lhook\joinrel\longrightarrow}\)} (B)
                    (E) -- node [sloped] {\({\longleftarrow\joinrel\rhook}\)} (B)
                    (E) -- node [sloped] {\({\lhook\joinrel\longrightarrow}\)} (C)
                    (D) -- node [sloped] {\({\longleftarrow\joinrel\rhook}\)} (C);
            \end{tikzpicture}
        \end{center}
        kommutiert.
    \end{definition}\noindent
    Ein zu \({\left(\mathcal{M}\times\mathbb{I},\mathcal{M},\mathcal{M}\right)}\) diffeomorpher Kobordismus hei\ss e trivial.
    \begin{definition}[H-Kobordismus]\label{def:h_cobordism}
        Ein Kobordismus \({\left(\mathcal{W},\mathcal{M},\mathcal{N}\right)}\) dessen Einbettungen \({\mathcal{M}\hookrightarrow\mathcal{W}}\) und \({\mathcal{N}\hookrightarrow\mathcal{W}}\) Homotopie\"aquivalenzen sind.
    \end{definition}
    
\section{Isotopien und Tubenumgebungen}\label{sec:isotopy}
    \begin{definition}[Isotopie]\label{def:isotopy}
        Eine Isotopie zweier Einbettungen \({\Psi,\Phi\colon\mathcal{M}\to\mathcal{N}}\) sei eine glatte Abbildung \({I\colon\mathcal{M}\times\mathbb{I}\to\mathcal{N}}\), sodass die Abbildungen
        \[I_t\colon\mathcal{M}\to\mathcal{N},\,p\mapsto I(p,t)\]
        f\"ur alle \({t\in\mathbb{I}}\) Einbettungen seien und \({I_0\equiv\Psi}\) sowie \({I_1\equiv\Phi}\) gelte. Dies sei durch \({\Psi\simeq\Phi}\) bezeichnet.
    \end{definition}
    Eine Isotopie bei welcher \({\mathcal{M}=\mathcal{N}}\) gilt, alle \({I_t}\) Diffeomorphismen seien und \({I_0\equiv\mathbbm{1}}\) gelte hei\ss e Diffeotopie. Ein wichtiger Begriff ist der der Tubenumgebung.
    \begin{definition}[Tubenumgebung von \({\mathcal{V}\subseteq\mathcal{M}}\) in \({\mathcal{M}}\) f\"ur \({\partial\mathcal{M}=\varnothing}\)]
        Eine Einbettung \({h\colon E\hookrightarrow\mathcal{M}}\) f\"ur ein Vektorb\"undel \({\pi\colon E\to\mathcal{N}^k}\) des Ranges \({n-k}\), sodass der Nullschnitt auf \({\mathcal{V}^k}\) abgebildet wird.
    \end{definition}
    Es l\"asst sich zeigen, dass \({\pi}\) in diesem Fall bereits zu dem Normalenb\"undel isomorph ist. Zu einer weiteren Tubenumgebung \({h^{\prime}\colon E^{\prime}\hookrightarrow\mathcal{M}}\) mit dem Vektorb\"undel \({\pi^{\prime}\colon E^{\prime}\to\mathcal{V}}\) existiert stets ein Isomorphismus \({\Gamma\colon\pi\to\pi^{\prime}}\), sodass \({h^{\prime}\simeq h\circ\Gamma}\) gilt. Sind \({\pi}\) und \({\pi^{\prime}}\) mit riemannsch, kann \({\Gamma}\) als Isometrie angenommen werden.
    \begin{definition}[Tubenumgebung einer Mannigfaltigkeit \({\mathcal{V}\subseteq\partial\mathcal{M}}\) in \({\mathcal{M}}\)]
        Die Fortsetzung einer Tubenumgebung \({h\colon E\hookrightarrow\partial\mathcal{M}}\) von \(\mathcal{V}\) zu einer Einbettung \({\Tilde{h}\colon E\times\mathbb{R}_{\geq0}\hookrightarrow\mathcal{M}}\).
    \end{definition}


\section{Nicht triviale wichtige S\"atze}
    Zun\"achst ist es elementar, Isotopien von Einbettungen in den Rand einer Mannigfaltigkeit auf die gesamte Mannigfaltigkeit fortzusetzen, sodass eine Isotopie tats\"achlich einen Diffeomorphismus induziert.
    \begin{proposition}\label{prop:isotopy_extension}
        F\"ur jede Isotopie \({I\colon\mathcal{M}\times\mathbb{I}\to\partial\mathcal{N}}\) existiert eine Diffeotopie \({J\colon\mathcal{N}\times\mathbb{I}\to\mathcal{N}}\) mit kompaktem Tr\"ager und \({J_t\circ I_0\equiv I_t}\) f\"ur alle \({t\in\mathbb{I}}\).
    \end{proposition}
    \begin{proof}
        Siehe \cite{hirsch2012difftop} Kapitel 8 Satz 1.3.
        \renewcommand\qedsymbol{\({\cancel{q.e.d.}}\)}
    \end{proof}
    
    \begin{proposition}[Schwacher Einbettungssatz von Whitney]\label{prop:whitney_weak_embedding}
        F\"ur \({n>2m}\) l\"asst sich jede stetige Abbildung \({\mathcal{M}^m\to\mathcal{N}^n}\) durch eine Einbettung approximieren.
    \end{proposition}
    \begin{proof}
        Siehe \cite{lee2013introduction} Satz 6.21.
        \renewcommand\qedsymbol{\({\cancel{q.e.d.}}\)}
    \end{proof}
    
    \begin{proposition}[Eindeutigkeit einer Einbettung]\label{prop:whitney_unique_embedding}
        Ist \({m\geq2n+2}\), so sind zwei homotope Einbettungen \({f,g\colon\mathcal{M}^n\to\mathcal{N}^m}\) isotop.
    \end{proposition}
    \begin{proof}
        Siehe \cite{whitney1936differentiable} Satz 6.
        \renewcommand\qedsymbol{\({\cancel{q.e.d.}}\)}
    \end{proof}

\newpage
\section{Verklebung von Mannigfaltigkeiten}
    \subsection{Die Randsumme}\label{subsec:boundary_connected_sum}
        Seien f\"ur \({i\in\{1,2\}}\) jeweils \({\mathcal{M}_i^n}\) Mannigfaltigkeiten, \({\mathcal{V}_i^k\subseteq\partial\mathcal{M}_i}\) Untermannigfaltigkeiten, \({\pi\colon E\to\mathcal{N}}\) ein riemannsches Vektorb\"undel und \({h_i\colon E\hookrightarrow\partial\mathcal{M}_i}\) Tubenumgebungen der \({\mathcal{V}_i}\) in \({\partial\mathcal{M}_i}\). Seien \({\Tilde{h}_i\colon E\times\mathbb{R}_{\geq0}\hookrightarrow\mathcal{M}_i}\) Fortsetzungen der \({h_i}\) zu Tubenumgebungen der \({\mathcal{V}_i}\) in \({\mathcal{M}_i}\). Sei zuletzt die Involution
        \[\alpha_E\colon E\times\mathbb{R}_{\geq0}\setminus\mathbf{0}\to E\times\mathbb{R}_{\geq0}\setminus\mathbf{0},\,v\mapsto\frac{v}{\norm{v}^2}\]
        gegeben, so setze
        \[\mathcal{M}_1\mathop{+}^{\mathcal{V}_1}\mathcal{M}_2=\left(\mathcal{M}_1\setminus\mathcal{V}_1\sqcup\mathcal{M}_2\setminus\mathcal{V}_2\right)/\left(\Tilde{h}_2(x)\sim\Tilde{h}_1\circ\alpha_E(x),x\in E\times\mathbb{R}_{\geq0}\setminus\mathbf{0}\right)\]
        oder falls die Wahl der \(h_i\) wichtig ist auch
        \[\mathcal{M}_1\mathop{+}_{h_1}^{h_2}\mathcal{M}_2\,.\]
        Diese Konstruktion ist bis auf Diffeomorphie unabh\"angig von der Wahl der Fortsetzung der \({h_i}\). Weiter ergeben isotope Einbettungen diffeomorphe Mannigfaltigkeiten. Die Notwendigkeit der Abbildung \(\alpha_E\) ist in Abbildung \ref{fig:need_for_alpha} illustriert.

        \begin{figure}
            \centering
            \begin{tikzpicture}
                \draw [red] 
                    (0.4, 1) -- (0, 1) 
                    (0.4, 1.5) -- (0, 1.5)
                    (0.4, -1) -- (0, -1)
                    (0.4, -1.5) -- (0, -1.5) 
                ;
                \draw 
                    (0.4, 1) node {\tiny\({[}\)} (0, 1) node {\tiny\({]}\)} 
                    arc (90:270:1) node [pos = 0.5, left] {\({\mathcal{N}}\)}
                    node {\tiny\({]}\)} (0.4, -1) node {\tiny\({[}\)}
                    (0, -1.5) node {\tiny\({]}\)} (0.4, -1.5) node {\tiny\({[}\)} 
                    arc (-90:90:1.5) node [pos = 0.5, right] {\({\mathcal{M}}\)}
                    node {\tiny\({[}\)} (0, 1.5) node {\tiny\({]}\)}
                    ;
                \draw [densely dotted] 
                    (0.4, 1) -- (0.4, 1.5)
                    (0.4, -1) -- (0.4, -1.5)
                    (0, 1) -- (0, 1.5)
                    (0, -1) -- (0, -1.5)
                ;
            \end{tikzpicture}
            \caption{Die Notwendigkeit von \({\alpha_E}\) in der Verklebung zweier eindimensionaler Mannigfaltigkeiten entlang ihres Randes.}\label{fig:need_for_alpha}
        \end{figure}
        \newpage
        F\"ur ein riemannsches Vektorb\"undel hei\ss e im Folgenden das glatte Faserb\"undel der Vektoren mit Norm \({\norm{v}\leq1}\) das Einheitsscheibenb\"undel. Die Faser ist also diffeomorph zu \({\mathbb{D}^n}\).
        \begin{lemma}[Verklebung mit einem Scheibenb\"undel]\label{lem:glueing_disc_bundles}
            Sei \({\pi\colon E\to\mathcal{V}\subseteq\mathcal{M}}\) ein riemannsches Vektorb\"undel vom Rang \({n-k}\), \({\mathcal{B}}\) das zugeh\"orige Einheitsscheibenb\"undel und \({s\colon\mathcal{V}\to\mathcal{V}^{\prime}\subseteq\partial\mathcal{B}}\) ein glatter Schnitt. Dann gilt
            \[\mathcal{M}\mathop{+}^{\mathcal{V}}\mathcal{B}\cong\mathcal{M}\,.\]
        \end{lemma}
        \begin{proof}
            Sei \({\left(\mathcal{V}^{\prime}\right)^{\perp}}\) das Komplement\"arb\"undel des von \({\mathcal{V}^{\prime}}\) in \({E}\) erzeugten Unterb\"undels. Dann ist
            \[h_1\colon\left(\mathcal{V}^{\prime}\right)^{\perp}\hookrightarrow\partial\mathcal{B},\,(p,v)\mapsto\frac{2v+\left(1-\norm{v}^2\right)p}{\norm{v}^2+1}\]
            eine Tubenumgebung von \({\mathcal{V}^{\prime}}\) in \({\partial\mathcal{B}}\) und
            \[\Tilde{h}_1\colon\left(\mathcal{V}^{\prime}\right)^{\perp}\times\mathbb{R}_{\geq0}\hookrightarrow\mathcal{B},\,(p,v,t)\mapsto\frac{2v+\left(1-t^2-\norm{v}^2\right)p}{\norm{v}^2+(1+t)^2}\]
            eine Tubenumgebung von \({\mathcal{V}^{\prime}}\) in \({\mathcal{B}}\). Seien weiter
            \[h_2\colon\left(\mathcal{V}^{\prime}\right)^{\perp}\hookrightarrow\partial\mathcal{M}\quad\text{und}\quad\Tilde{h}_2\colon\left(\mathcal{V}^{\prime}\right)^{\perp}\times\mathbb{R}_{\geq0}\hookrightarrow\mathcal{M}\]
            beliebige Tubenumgebungen von \({\mathcal{V}}\) in \({\partial\mathcal{M}}\) und \({\mathcal{M}}\). Dann l\"asst sich der Diffeomorphismus
            \[\Phi\colon\mathcal{M}\mathop{+}^{\mathcal{V}}\mathcal{B}\to\mathcal{M},\,z\mapsto\begin{cases}
                 z & z\in\mathcal{M}\setminus\mathcal{V}\\
                \Tilde{h}_2\left(\frac{2v+\left(1-\norm{v}^2-t^2\right)p}{\norm{v}^2+\left(1-t\right)^2}\right) & z=v+tp\in\mathcal{B}\setminus\mathcal{V}^{\prime}
            \end{cases}\]
            definieren. Es l\"asst sich leicht (?) nachrechnen, dass dieser wohldefiniert ist (siehe Appendix \ref{app:diff_well_defined}). Dies zeigt die Aussage f\"ur die explizite Wahl der Tubenumgebung \({h_1}\) und beliebige Tubenumgebungen \({h_2}\). Da f\"ur jede andere Tubenumgebung \({h\colon\left(\mathcal{V}^{\prime}\right)^{\perp}\to\partial\mathcal{B}}\) eine Isometrie \({\Gamma}\) existiert, sodass \({h\simeq h_1\circ\Gamma}\) gilt, folgt dann aber bereits
            \[\mathcal{M}\mathop{+}_h^{h_2}\mathcal{B}\cong\mathcal{M}\mathop{+}_{h_1\circ\Gamma}^{h_2}\mathcal{B}=\mathcal{M}\hspace{-8pt}\mathop{+}_{h_1}^{\hspace{5pt}h_2\circ\Gamma^{-1}}\hspace{-9pt}\mathcal{B}\cong\mathcal{M}\,,\]
            gem\"a\ss{} dem oben Gezeigten, und da \({\alpha}\) mit Isometrien kommutiert.
        \end{proof}


    \newpage % -- Ausbessern --
    \chapter{Henkelk\"orper}\label{chp:handlebodies}
        \section{Anbringen von Henkeln}\label{sec:handle_attachment}
    Im Folgenden hei\ss e \({(x,y)\in\mathbb{D}^n}\) stets \({x\in\mathbb{D}^k}\) und \({y\in\mathbb{D}^k}\). Weiter sei zum Beispiel \({\mathbb{S}^{k-1}}\) durch \({\mathbb{S}^{k-1}\times\{0\}^{n-k}}\) als Untermannigfaltigkeit von \({\partial\mathbb{D}^n}\) aufgefasst.
    \subsection{Mannigfaltigkeiten mit Ecken}\label{subsec:manifolds_with_corners}
        Sei \({\mathcal{W}}\) eine glatte Mannigfaltigkeit mit Rand. Es kann die Mannigfaltigkeit mit Ecken \({H^k:=\mathbb{D}^k\times\mathbb{D}^{n-k}}\) betrachtet werden. Ihr Rand ist gerade
        \[\partial H^k=\mathbb{S}^{k-1}\times\mathbb{D}^{n-k}\cup\mathbb{D}^k\times\mathbb{S}^{n-k-1}\,,\]
        und l\"asst sich entlang einer Einbettung \({\Psi^k\colon\mathbb{S}^{k-1}\times\mathbb{D}^{n-k}\to\partial\mathcal{W}}\) zu dem topologischen Raum
        \[\mathcal{W}\mathop{+}^{\Psi}H^k:=\left(\mathcal{W}\sqcup H^k\right)/\left(\forall p\in\mathbb{S}^{k-1}\times\mathbb{D}^{n-k}\colon p\sim\Psi(p)\right)\]
        verkleben. Dieser Raum ist zweitabz\"ahlbar, hausdorffsch und lokal euklidisch. Das Problem besteht darin, dass es sich hierbei nicht unbedingt um eine glatte Mannigfaltigkeit mit Rand handelt, sondern vielmehr um eine glatte Mannigfaltigkeit mit Ecken, was den gesamten Beweis in die Kategorie der Mannigfaltigkeiten mit Ecken verschiebt, was etwas l\"astig ist. Es ist m\"oglich, diese Ecken zu gl\"atten, also eine hom\"oomorphe glatte Mannigfaltigkeit mit Rand zu finden. Dies ist nicht trivial und die Existenz einer Gl\"attung m\"usste gezeigt werden, sowie dieser Umstand bei allen Beweisen beachtet werden. Abgesehen davon ist der Ansatz etwas intuitiver und in Abbildung \ref{fig:one_handle} dargestellt.
        
    \newpage
    \subsection{Randsumme mit der Einheitsscheibe}\label{subsec:boundary_connected_sum_with_unit_disc}
        Andererseits l\"asst sich das Anbringen eines Henkels als die Randsumme einer Mannigfaltigkeit mit der Einheitsscheibe beschreiben. Hierzu sei die Umgebung \({U:=\mathbb{D}^n\setminus\mathbb{D}^{n-k}}\) von \({\mathbb{S}^{k-1}}\) in \({\mathbb{D}^n}\) definiert. Diese ist das Bild einer geeigneten Tubenumgebung \({\Tilde{h}_1\colon E\times\mathbb{R}_{\geq0}\hookrightarrow\mathbb{D}^n}\) mit dem riemannschen Vektorb\"undel
        \[\pi\colon E\to\mathbb{S}^{k-1},\,(p,x)\mapsto p\]
        f\"ur \({E:=\mathbb{S}^{k-1}\times\mathbb{R}^{n-k}}\) (siehe Appendix \ref{app:sphere_tub_emb}). Unter dieser Einbettung ist \({\alpha:=\Tilde{h}_1\circ\alpha_E\circ\Tilde{h}_1^{-1}}\) gerade
        \[\alpha\colon U\setminus\mathbb{S}^{k-1}\to U\setminus\mathbb{S}^{k-1},\,(x,y)\mapsto\left(x\frac{\sqrt{1-\norm{x}^2}}{\norm{x}},y\frac{\norm{x}}{\sqrt{1-\norm{x}^2}}\right)\,.\]
        Die Einschr\"ankung von \(\Tilde{h}\) auf \(E\times\{0\}\) ein, ergibt eine Tubenumgebung \(h\) von \({\mathbb{S}^{k-1}}\) in \({\partial\mathbb{D}^n}\) mit dem Bild \({U\cap\partial\mathbb{D}^n}\). Sei eine Einbettung \({\psi^k\colon\mathbb{S}^{k-1}\hookrightarrow\partial\mathcal{W}}\) mit einer Fortsetzungen \({\Psi^k\colon U\cap\partial\mathbb{D}^n\hookrightarrow\partial\mathcal{W}}\) und einer weiteren Fortsetzung \({\Tilde{\Psi}^k\colon U\hookrightarrow\mathcal{W}}\) von \({\Psi^k}\) gegeben, kann die Tubenumgebung \({\Tilde{h}_2:=\Tilde{\Psi}\circ\Tilde{h}_1}\) von \({\Lambda^k:=\psi^k(\mathbb{S}^{k-1})}\) in \({\mathcal{W}}\), und somit auch
        \[\mathcal{W}+\Psi^k:=\mathcal{W}\mathop{+}^{\mathbb{S}^{k-1}}\mathbb{D}^n\]
        definiert werden. In dieser gilt die Identifikation \({\Tilde{h}_1(x)\sim\Tilde{h}_2\circ\alpha_E(x)}\) f\"ur alle \({x\in E\times\mathbb{R}_{\geq0}\setminus\mathbf{0}}\), und wegen
        \[\Tilde{h}_2\circ\alpha_E(x)=\Tilde{\Psi}\circ\Tilde{h}_1\circ\Tilde{h}_1^{-1}\circ\alpha\circ\Tilde{h}_1(x)=\Tilde{\Psi}\circ\alpha\circ\Tilde{h}_1(x)\]
        auch \({y\sim\Tilde{\Psi}\circ\alpha(y)}\) f\"ur \({y\in U\setminus\mathbb{S}^{k-1}}\). Folglich ist
        \[\mathcal{W}+\Psi^k=\left(\mathcal{W}\setminus\Lambda^k\sqcup\mathbb{D}^n\setminus\mathbb{S}^{k-1}\right)/\left(y\sim\Tilde{\Psi}\circ\alpha(y)\,,y\in U\setminus\mathbb{S}^{k-1}\right)\]
        und es muss nicht mehr mit den tats\"achlichen Tubenumgebungen gearbeitet werden. Ein Henkel sei das Tripel \({(\psi^k,\Psi^k,\Tilde{\Psi}^k)}\), welches auch mit \({\Psi^k}\) bezeichnet werde.
    
    \subsection{\"Aquivalenz der Ans\"atze}\label{rem:equiv_corners_boundary_sum}
        Die R\"aume aus den Abschnitten \ref{subsec:manifolds_with_corners} und \ref{subsec:boundary_connected_sum} sind zueinander ho\-m\"oo\-morph, sodass die zusammenh\"angende Summe entlang der eingebetteten Sph\"are im Rand als Gl\"at\-tung der oben diskutierten Mannigfaltigkeit mit Ecken angesehen werden kann. 
    
    \subsubsection{Notation}
        Sei \({\Psi^k}\) ein Henkel, so hei\ss e die durch \({\Psi^k}\) eingebettete Sph\"are \({\Lambda^k}\) die Anklebesph\"are von \({\Psi^k}\) und \({\Sigma^k:=\mathbb{S}^{n-k-1}}\) der G\"urtel von \({\Psi^k}\). Hierbei gilt die Beziehung \({\Sigma^k=\partial\left(\mathcal{W}+\Psi^k\right)\setminus\partial\mathcal{W}}\). Der Kern von \({\Psi^k}\) sei die abgeschlossene H\"ulle von \({\mathring{\mathbb{D}}^{n-k}}\) in \({\mathcal{W}+\Psi^k}\).
        
        \begin{figure}
            \centering
            \begin{tikzpicture}
                \begin{scope}[yshift = 3cm] % Small ball
                    \draw [red, densely dotted]
                        (-90:0.5 and 1) arc (-90:90:0.5 and 1);
                    \draw
                        (1, 0) arc (0:360:1)
                        (180:1 and 0.5) arc (180:360:1 and 0.5);
                    \draw [densely dotted]
                        (0:1) arc (0:180:1 and 0.5);
                    \draw [fill = blue, opacity = 0.2] 
                        (1, 0) arc (0:360:1) -- cycle;
                    \draw [red, thick]
                        (90:0.5 and 1) arc (90:270:0.5 and 1);
                \end{scope}
                 %  Positions of circles
                \path[scale = 3] (-50:1) arc (-50:230:1)
                \foreach\i in {0, 0.05, 0.1, 0.15, 0.2, 0.25, 0.3, 0.35, 0.4, 0.45, 0.5, 0.55, 0.6, 0.65, 0.7, 0.75, 0.8, 0.85, 0.9, 0.95, 1} { node [pos = \i] (A\i) {}};
                % End of Tube
                \draw [thick]
                    (-41.75:4) arc (-41.75:230:4) 
                    (-19.5:2) arc (-19.5:230:2);
                \draw [densely dotted] 
                    (-50:4) arc (-50:-41.75:4) 
                    (-50:2) arc (-50:-19.5:2);
                % First circle
                \draw [densely dotted, rotate = -50] ({A0}.center) ++(0:1 and 0.5) arc (0:360:1 and 0.5);
                \begin{scope}[rotate = -36]     % Second circle
                    \draw [thick] ({A0.05}.center) ++(0:1 and 0.5) arc (0:114:1 and 0.5);
                    \draw [densely dotted] ({A0.05}.center) ++(115:1 and 0.5) arc (115:180:1 and 0.5);
                    \draw [loosely dashdotted] ({A0.05}.center) ++(-180:1 and 0.5) arc (-180:-32:1 and 0.5);
                    \draw [dashed] ({A0.05}.center) ++(-32:1 and 0.5) arc (-32:0:1 and 0.5);
                \end{scope}
                \begin{scope}[rotate = -22]     % Third circle
                    \draw [thick] ({A0.1}.center) ++(0:1 and 0.5) arc (0:170:1 and 0.5);
                    \draw [dashed] ({A0.1}.center) ++(-86:1 and 0.5) arc (-86:0:1 and 0.5);
                    \draw [densely dotted] ({A0.1}.center) ++(170:1 and 0.5) arc (170:180:1 and 0.5);
                    \draw [dashdotted] ({A0.1}.center) ++(180:1 and 0.5) arc (180:274:1 and 0.5);
                \end{scope}
                % Big ball
                \begin{scope}[yshift = -3cm, scale = 3]
                    \draw [thick]
                        (-186.5:1) arc (-186.5:129:1) 
                        (-90:0.5 and 1) arc (-90:90:0.5 and 1) 
                        (180:1 and 0.5) arc (180:360:1 and 0.5);
                    \draw [dashed]
                        (129:1) arc (129:173.5:1);
                    \draw [densely dotted]
                        (0:1 and 0.5) arc (0:114:1 and 0.5) 
                        (180:1 and 0.5) arc (180:169:1 and 0.5) 
                        (90:0.5 and 1) arc (90:150.5:0.5 and 1) 
                        (270:0.5 and 1) arc (270:171.25:0.5 and 1);
                    \draw [loosely dashdotted] 
                        (114:1 and 0.5) arc (114:169: 1 and 0.5)
                        (150.5:0.5 and 1) arc (150.5:171.25:0.5 and 1);
                \end{scope}
                % Remaining circles
                \foreach\i in {0.15, 0.2, 0.25, 0.3, 0.35, 0.4, 0.45, 0.55, 0.6, 0.65, 0.7, 0.75, 0.8, 0.85, 0.9, 0.95, 1} {
                    \begin{scope} [rotate = -50 + \i * 280]
                        \draw [thick] ({A\i}.center) ++(0:1 and 0.5) arc (0:180:1 and 0.5);
                        \draw [dashed] ({A\i}.center) ++(180:1 and 0.5) arc (180:360:1 and 0.5);
                    \end{scope}
                }
                % Attaching-Spheres
                \draw [fill = blue, opacity = 0.2, rotate = 230] 
                    ({A1}.center) ++(0:1 and 0.5) arc (0:360:1 and 0.5);
                \draw [fill = blue, opacity = 0.2, rotate = -50] 
                    ({A0}.center) ++(0:1 and 0.5) arc (0:360:1 and 0.5);
                \draw 
                    ({A1}.center) node [circle, fill, inner sep=0.75pt] {}
                    ({A0}.center) node [circle, fill, inner sep=0.75pt] {};
            \end{tikzpicture}
            \caption{Eine 3-Vollkugel mit einem 1-Henkel, die gleichzeitig eine optische T\"auschung ist. Dies ist hom\"oomorph zu einem Torus, also einem Diskb\"undel des Ranges \({2}\) \"uber der \({1}\)-Sph\"are (siehe Bemerkung \ref{rem:equiv_corners_boundary_sum} und Lemma \ref{lem:glueing_disc_bundles}).}\label{fig:one_handle}
        \end{figure}
    
    \newpage
    \begin{lemma}[Henkel an der Einheitskugel]\label{lem:handle_on_unit_disc}
        Das Anbringen eines Henkels an \({\mathbb{D}^n}\) entlang \({\mathbb{S}^{k-1}}\) ist zu einem Schei\-ben\-b\"un\-del des Ranges \({n-k}\) diffeomorph.
    \end{lemma}
    \begin{proof}
        Sei die Anklebeabbildung durch \({\psi^k\colon\mathbb{S}_1^{k-1}\to\mathbb{S}_2^{k-1}}\) gegeben. Eine m\"ogliche Fortsetzung von dieser ist
        \[\dot{\Psi}^k\colon U_1\cap\partial\mathbb{D}_1^n\to U_2\cap\partial\mathbb{D}_2^n,\,(x,y)\mapsto\left(\norm{x}\psi^k\left(\frac{x}{\norm{x}}\right),y\right)\,.\]
        Sei \({\ddot{\Psi}^k\colon U_1\cap\partial\mathbb{D}_1^n\to U_2\cap\partial\mathbb{D}_2^n}\) eine weitere, beliebige Fortsetzung von \({\psi^k}\). Dann sind die zugeh\"origen Tubenumgebungen von \({\mathbb{S}_2^{k-1}}\) in \({\partial\mathbb{D}_2^n}\) gerade
        \[\dot{h}_2=\dot{\Psi}^k\circ h_1\quad\text{und}\quad\ddot{h}_2=\ddot{\Psi}^k\circ h_1\,,\]
        und es existiert eine Isometrie, sodass \({\ddot{h}_2\simeq\dot{h}_2\circ\Gamma}\), und somit auch
        \[\ddot{\Psi}^k=\ddot{h}_2\circ h^{-1}\simeq\dot{h}_2\circ\Gamma\circ h_1^{-1}=\dot{\Psi}^k\circ h_1\circ\Gamma\circ h_1^{-1}\]
        gilt. Hierbei existiert eine glatte Abbildung \({\gamma\colon\mathbb{S}_1^{k-1}\to O\left(n-k\right)}\), sodass
        \[h_1\circ\Gamma\circ h_1^{-1}(x,y)=\left(x,\gamma\left(\frac{x}{\norm{x}}\right)\cdot y\right)\]
        ist. Folglich kann als Fortsetzung stets \({\Psi^k:=\dot{\Psi}^k\circ h_1\circ\Gamma\circ h_1^{-1}}\) gew\"ahlt werden, f\"ur welche
        \[\Psi(x,y)=\left(\norm{x}\psi^k\left(\frac{x}{\norm{x}}\right),\gamma\left(\frac{x}{\norm{x}}\right)\cdot y\right)\]
        gilt. Eine Fortsetzung dieser ist
        \[\Tilde{\Psi}^k\colon U_1\to U_2,\,z\mapsto\norm{z}\Psi^k\left(\frac{z}{\norm{z}}\right)\,.\]
        Diese Konstruktion ergibt nun
        \[\mathbb{D}^n+\Psi^k=\left(\mathbb{D}_1^n\setminus\mathbb{S}_1^{k-1}\right)\sqcup\left(\mathbb{D}_2^n\setminus\mathbb{S}_2^{k-1}\right)/\left(z\sim\Tilde{\Psi}^k\circ\alpha(z)\right)\]
        und besitzt die Untermannigfaltigkeit
        \[\mathcal{M}:=\mathring{\mathbb{D}}_1^k\sqcup\mathring{\mathbb{D}}_2^k/\left(x\sim\Tilde{\Psi}^k\circ\alpha(x)\right)\,,\]
        die zu der \({k}\)-Sph\"are hom\"oomorph ist (!). Es l\"asst sich nachrechnen, dass diese Identifikationen mit der Projektion
        \[\mathbb{D}^n+\Psi^k\to\mathcal{M},\,(x,y)\mapsto x\]
        kommutieren (siehe Appendix \ref{app:disc_bundle_well_defined}), sodass sich eine Faserb\"undelstruktur ergibt, deren Fasern diffeomorph zu Scheiben sind.
    \end{proof}

    \newpage
    \section{Vertauschung der Befestigungsreihenfolge}
        \begin{lemma}[Sortierungs-Lemma]\label{lem:sort_handles}
            Sei \({\mathcal{W}^n}\) ein Kobordismus und
            \[\Psi^i\colon\mathbb{S}^{i-1}\to\partial_+\mathcal{W}\quad\text{sowie}\quad\Psi^j\colon\mathbb{S}^{j-1}\to\partial_+\left(\mathcal{W}+\Psi^i\right)\]
            Henkel mit \({j\leq i\leq n}\). Dann existiert ein in \({\partial_+\left(\mathcal{W}+\Psi^i\right)}\) zu \({\Psi^j}\) iso\-to\-
            per Henkel \({\Phi^j\colon\mathbb{S}^{j-1}\to\partial_+\mathcal{W}}\) derart, dass
            \[\left(\mathcal{W}+\Psi^i\right)+\Psi^j\cong\left(\mathcal{W}+\Phi^j\right)+\Psi^i\,.\]
        \end{lemma}
        \begin{proof}
            Der Henkel \({\Psi^j}\) ist isotop zu einem Henkel dessen Anklebesph\"are \({\Lambda^j}\) den G\"urtel von \({\Psi^i}\) transversal schneide. Da die Anklebesph\"are die Dimension \({j-1}\), und der G\"urtel die Dimension \({n-i-1}\) besitzt, kann die Dimension der Summe der Tangentialvektorr\"aume in allen Schnittpunkten h\"ochstens
            \[(j-1)+(n-i-1)\leq n-2<n-1=\dim\partial_+\mathcal{W}\]
            betragen, sodass Transversalit\"at nur im trivialen Fall gegeben sein kann, also wenn der Schnitt leer ist. Ist die Anklebesph\"are jedoch disjunkt von dem G\"urtel von \({\Psi^i}\), so liegt diese per Definitionem bereits komplett in \({\partial_+\mathcal{W}}\). Jede Tubenumgebung von \({\Lambda^j}\) kann durch eine weitere Isotopie derart geschrumpft werden, dass diese ebenso komplett in \({\partial_+\mathcal{W}\setminus\Lambda^i}\) liegt. Das Anbringen des resultierenden Henkels \({\Phi^j}\) ist nun nicht mehr von \({\Psi^i}\) abh\"angig, sodass die Reihenfolge vertauscht werden kann.
        \end{proof}

        \subsection{Allgemeine Vertauschung}
            Sei \({\mathcal{W}}\) ein Kobordismus. Liegen f\"ur \({1\leq j\leq l}\) Einbettungen 
            \[\Psi_j^k\colon\mathbb{S}^{k-1}\to\partial_+\left(\mathcal{W}+\Psi_1^k+\Psi_2^k+\dots+\Psi_{j-1}^k\right)\] 
            vor, so zeigt eine zu dem Beweis des Lemmas sehr \"ahnliche Argumentation, dass f\"ur jedes \({\Psi_j^k}\) eine isotope Einbettung
            \[\Phi_j^k\colon\mathbb{S}^{k-1}\to\partial_+\mathcal{W}\]
            gefunden werden kann, deren Bild von allen anderen Anklebesph\"aren disjunkt ist. Insbesondere muss bis auf Diffeomorphie bei der Definition nicht auf die zuvor angebrachten Henkel geachtet werden. Dies rechtfertigt die Schreibweise
            \begin{align*}
                \mathcal{W}+\sum_{j=1}^l\Psi_j^k:&=\left(\dots\left(\left(\mathcal{W}+\Psi_1^k\right)+\Psi_2^k\right)+\dots\right)+\Psi_l^k\\
                &\cong\mathcal{W}+\Phi_1^k+\Phi_2^k+\dots+\Phi_l^k\,.
            \end{align*}

        \subsection{Henkelzerlegungen}
            Der Vertauschungssatz erm\"oglicht nun eine gezielte Definition einer Henkelzerlegung eines Kobordismus \({\left(\mathcal{W}^n,\mathcal{M},\mathcal{N}\right)}\).
            \begin{definition}[Henkelzerlegung]
                Eine Folge von Kobordismen \({\left(\mathcal{W}_k,\mathcal{M},\partial_+\mathcal{W}_k\right)}\) mit \({\mathcal{W}_{-1}=\mathcal{M}\times\mathbb{I}}\), \({\mathcal{W}_n\cong\mathcal{W}}\) und Zahlen \({p_k\in\mathbb{N}}\) f\"ur \({0\leq k\leq n}\), sowie Henkeln 
                \[\Phi_j^{k+1}\colon\mathbb{S}^k\to\partial_+\mathcal{W}_k\,,\quad\text{sodass}\quad\mathcal{W}_{k+1}\cong\mathcal{W}_k+\sum_{j=1}^{p_k}\Phi_j^{k+1}\]
                gilt und die Anklebesph\"aren aller \({\Phi_j^{k+1}}\) voneinander disjunkt seien.
            \end{definition}
            In der Anwendung des Sortierungs-Lemmas auf Henkel in einer Henkelzerlegung sei Obacht geboten. Durch die Diffeotopie werden auch m\"og\-lich\-er\-wei\-se die Anklebesph\"aren von sp\"ater angebrachten Henkeln verschoben.

    \newpage % -- Ausbessern --
    \chapter{Etwas Morsetheorie}\label{chp:morse_theory}
        \section{Morsefunktionen}
    Ein kritischer Punkt \({p\in\mathcal{W}}\) einer glatten Funktion \({f\colon\mathcal{W}\to[0,1]}\) hei\ss e entartet, falls die Matrix der
    \[\frac{\partial^2f}{\partial\alpha_i\partial\alpha_j}\bigg|_p=\frac{\partial}{\partial\alpha_i}\left(q\mapsto\frac{\partial f}{\partial\alpha_j}\bigg|_q\right)\bigg|_p\in\mathbb{R}\]
    singul\"ar ist.
    \begin{definition}[Morse-Funktion]
        Eine differenzierbare Abbildung \({f\colon\mathcal{W}\to[0,1]}\), sodass alle kritischen Punkte weder auf dem Rand von \({\mathcal{M}}\) liegen noch entartet sind, und \({f^{-1}\left(\{0\}\right)=\mathcal{M}}\) und \({f^{-1}\left(\{1\}\right)=\mathcal{N}}\) gelte.
    \end{definition}
    Die Bedeutung von Morse-Funktionen sei hier nur angerissen und es werden keine Beweise gegeben. F\"ur weitere Informationen siehe auch \cite{kosinski2013differential} Kapitel 7.

\section{Existenz einer Henkelzerlegung}
    Auf jedem Kobordismus \({\left(\mathcal{W},\mathcal{M},\mathcal{N}\right)}\) existiert eine Morse-Funktion \({f}\) mit kritischen Werten \({0<c_1<\dots<c_l<1}\). Seien \({d_k\in\,[0,1]}\) regul\"are Werte, sodass
    \[0=d_0<c_1<d_1<\dots<d_{l-1}<c_l<d_l=1\]
    gilt. Dann ist \({\mathcal{M}_j:=f^{-1}\left(\{d_j\}\right)}\) f\"ur \({0\leq j\leq l}\) eine geschlossene Mannigfaltigkeit, und \({\mathcal{F}_j:=f^{-1}\left([d_j,d_{j+1}]\right)}\) f\"ur \({0\leq j\leq l-1}\) ein Kobordismus von \({\mathcal{M}_j}\) zu \({\mathcal{M}_{j+1}}\). Aus 
    \[\mathcal{F}_j\mathop{+}^{\mathcal{M}_{j+1}}\mathcal{F}_{j+1}\cong f^{-1}\left(\left[d_j,d_{j+2}\right]\right)\]
    folgt induktiv
    \[\mathcal{W}\cong\mathcal{F}_0\mathop{+}^{\mathcal{M}_1}\mathcal{F}_1\mathop{+}^{\mathcal{M}_2}\dots\mathop{+}^{\mathcal{M}_{l-2}}\mathcal{F}_{l-2}\mathop{+}^{\mathcal{M}_{l-1}}\mathcal{F}_{l-1}\,.\]
    Es l\"asst sich zeigen, dass f\"ur alle \({j}\) ein \({i_j}\) und ein Henkel \({\Psi^{i_j}\colon\mathbb{S}^{i_j-1}\to\mathcal{M}_{j+1}}\) existiert, sodass
    \[\mathcal{F}_j\cong\left(\mathcal{M}_j\times\mathbb{I}\right)+\Psi^{i_j}\]
    gilt. Es folgt erneut induktiv, dass
    \[\mathcal{W}\cong\mathcal{M}\times\mathbb{I}+\sum_{j=1}^l\Psi_j^{i_j}\]
    gilt. Durch Sammeln aller \({k}\)-Henkel mithilfe des Sortierungssatzes ergibt sich eine gew\"unschte Henkeldarstellung.

    \subsubsection{Duale Repr\"asentation}
        Die Funktion \({1-f}\) ergibt einen Kobordismus von \({\mathcal{N}}\) zu \({\mathcal{M}}\) und eine Henkelzerlegung, in der jeder \({k}\)-Henkel der originalen Zerlegung mit genau einem \({n-k}\) Henkel korrespondiert.
    
    \newpage % -- Aufräumen --
    \chapter{Einfache K\"urzungen}\label{chp:handle_cancellation}
        \begin{lemma}\label{lem:smooth_section}
    Schneidet die Untermannigfaltigkeit \({\mathcal{V}}\) eines glatten Faserb\"undels \({\pi\colon\mathcal{B}\to\mathcal{N}}\) jede Faser transversal und in genau einem Punkt, ist \({\mathcal{V}}\) ein glatter Schnitt.
\end{lemma}
\begin{proof}
    Da jede Faser in genau einem Punkt geschnitten wird, kann eine zu \({\pi}\) rechtsinverse Abbildung \({s\colon\mathcal{N}\to\mathcal{B}}\) definiert werden. Es verbleibt zu zeigen, dass \({s}\) glatt ist. Aus der Transversalit\"atsbedingung folgt, dass \({T_p(\pi|_{\mathcal{V}})}\) in jedem Punkt \({p\in\mathcal{V}}\) surjektiv (!), also ein Isomorphismus ist. Dann zeigt jedoch der Satz \"uber die implizite Funktion, dass die lokale Umkehrfunktion, also \({s}\), in \({p}\) glatt ist.
\end{proof}

\begin{theorem}[Aufhebungssatz]\label{thm:handle_removement}
    Schneidet die Anklebesph\"are eines Henkels \({\Psi^{k+1}}\) den G\"urtel eines Henkels \({\Psi^k\colon\mathbb{S}^{k-1}\to\partial_+\mathcal{W}}\) transversal und in genau einem Punkt, ist 
    \[\mathcal{W}\cong\mathcal{W}+\Psi^k+\Psi^{k+1}\,.\]
\end{theorem}
\begin{proof}
    Zun\"achst existiert ein Diffeomorphismus \({\displaystyle\mathcal{W}\cong\mathcal{W}\mathop{+}^*\mathbb{D}^n}\) f\"ur einen Punkt \({*\in\partial_+\mathcal{W}}\). Da die Untermannigfaltigkeit \({\Lambda^{k+1}\setminus\{x\}\cup\Lambda^k\subseteq\partial_+\mathcal{W}}\) eine Scheibe ist (!), kann eine Diffeotopie gefunden werden, die diese auf die Scheibe \({\mathbb{S}_{\geq0}^k\subseteq\mathbb{D}^n}\), also \({\Lambda^k}\) auf \({\mathbb{S}^{k-1}\subseteq\mathbb{D}^n}\) abbildet. Gem\"a\ss{} Satz \ref{lem:handle_on_unit_disc} existiert eine Scheibenb\"undelstruktur \({\pi\colon\mathbb{D}^n+\Psi^k\to\mathcal{M}}\), wobei \({\Lambda^{k+1}}\) in \({\partial_+(\mathcal{W}+\Psi^k)}\) auf \({\mathbb{S}_{>0}^k\cup\{x\}}\) abgebildet wird und aufgrund der Annahme alle Fasern von \({\pi}\) transversal und in genau einem Punkt schneidet. Dies ist gem\"a\ss{} Satz \ref{lem:smooth_section} ein glatter Schnitt. Es folgt
    \begin{align*}
        \mathcal{W}+\Psi^k+\Psi^{k+1}\cong\mathcal{W}\mathop{+}^*\left(\mathbb{D}^n+\dot\Psi^k\right)+\dot\Psi^{k+1}\mathop{\cong}^{\ref{lem:glueing_disc_bundles}}\mathcal{W}\mathop{+}^*\mathbb{D}^n\cong\mathcal{W}\,.
    \end{align*}

\end{proof}
\newpage
Ein Henkel hei\ss e trivial, falls die Anklebeabbildung durch \({\mathbb{D}^{n-1}}\) faktorisiere. In diesem Fall l\"asst sich ein additives Rechtsinverses explizit konstruieren, was hier jedoch vermieden werde.
\begin{corollary}[Ersetzungssatz]\label{cor:handle_replacement}
    Sei \({1\leq k\leq n-3}\). Die Anklebeabbildung eines Henkels \({\Psi^{k+1}}\) schneide als einzigen \({k}\)-G\"urtel den G\"urtel von \({\Psi_{c_k}^k}\) transversal und in genau einem Punkt, und sei in \({\partial_+\mathcal{W}_{k+1}}\) isotop zu einem trivialen Henkel \({\Phi^{k+1}}\). Dann existiert eine Henkelzerlegung mit \({c_k-1}\) \({k}\)-Henkeln.
\end{corollary}
\begin{proof}
    Sei ein zu \({\Phi^{k+1}}\) additiv rechtsinverser Henkel durch \({\Psi^{k+2}}\) gegeben. Dann gilt
    \begin{align*}
        \mathcal{W}_{k+1}&\cong\mathcal{W}_{k+1}+\Phi^{k+1}+\Psi^{k+2}\\
        &\cong\mathcal{W}_{k+1}+\Psi^{k+1}+\dot{\Psi}^{k+2}\\
        &\hspace{-3pt}\mathop{\cong}^{\text{Def.}}\mathcal{W}_{k-1}+\sum_{j=1}^{c_k}\Psi_j^k+\sum_{j=1}^{c_{k+1}}\Psi_j^{k+1}+\Psi^{k+1}+\dot{\Psi}^{k+2}\\
        &\mathop{\cong}^{\ref{lem:sort_handles}}\mathcal{W}_{k-1}+\sum_{j=1}^{c_k-1}\Psi_j^k+\Psi_{c_k}^k+\Psi^{k+1}+\sum_{j=1}^{c_{k+1}}\dot{\Psi}_j^{k+1}+\ddot{\Psi}^{k+2}\\
        &\mathop{\cong}^{\ref{thm:handle_removement}}\mathcal{W}_{k-1}+\sum_{j=1}^{c_k-1}\Psi_j^k+\sum_{j=1}^{c_{k+1}}\ddot{\Psi}_j^{k+1}+\dot{\ddot{\Psi}}^{k+2}\,,
    \end{align*}
    wobei die Akzente implizieren, dass es sich technisch gesehen nicht mehr um die gleichen Henkel handele.
\end{proof}

        
    \newpage % -- Modifikationslemma unvollständig --
    \chapter{Homologie von Kobordismen}\label{chp:cobordism_homology}
        \section{Henkul\"are Homologie}
    Aus homologischer Sicht, ist die Henkelzerlegung sehr \"ahnlich zu der CW-Struktur eines zellul\"aren Komplexes, auch wenn die Ans\"atze nicht identisch sind. Die Konstruktion der Verklebung eines Henkels als Randsumme mit der Einheitskugel ist bei diesen \"Uberlegungen nachteilhaft, da dort keine kanonischen Inklusionen \(\mathcal{W},\mathbb{D}^n\hookrightarrow\mathcal{W}+\Psi^k\) existieren. Folglich sei \(\mathcal{W}+\Psi^k\) im Folgenden als Summe von \(\mathcal{W}\) mit \(\mathbb{D}^k\times\mathbb{D}^{n-k}\) verstanden. Dies ist m\"oglich da die beiden Ans\"atze hom\"oomorphe Mannigfaltigkeiten liefern. Folgende Konstruktion verl\"auft nahezu analog zu \cite{hatcher2002algebraic} [Section 2.2, Cellular Homology]. Der einzige Unterschied besteht darin, dass die Henkelzerlegung nicht von der leeren Menge ausgehend konstruiert wird, sodass statt der langen exakten Folge der Paare \(\left(\mathcal{W}_k,\mathcal{W}_{k-1}\right)\) eine lange exakte Folge von Tripeln \(\left(\mathcal{W}_k,\mathcal{W}_{k-1},\mathcal{M}\times\mathbb{I}\right)\) betrachtet werden muss. Es gelten folgende Eigenschaften (Vergleiche Hatcher Section 2.2 Lemma 2.34)
    \begin{lemma}\label{lem:handular_homology}
        Ist \(\mathcal{W}\) ein Kobordismus mit Henkelzerlegung \(\mathcal{W}_k\), so gilt:
        \begin{enumerate}
            \item[i)] \(H_k\left(\mathcal{W}_m,\mathcal{W}_{m-1}\right)\not=0\) genau dann, wenn \(m=k\) ist. Dann ist diese frei abelsch und von den Kernen der \(k\)-Henkel erzeugt.
            \item[ii)] Die Abbildung \(H_k\left(\mathcal{W}_m,\mathcal{M}\right)\to H_k\left(\mathcal{W},\mathcal{M}\right)\) ist ein Isomorphismus f\"ur \(k<m\) und ein Epimorphismus f\"ur \(k=m\).
            \item[iii)] \(H_k\left(\mathcal{W}_m,\mathcal{M}\right)=0\) f\"ur \(k>m\).
        \end{enumerate}
    \end{lemma}
    \begin{proof}
        \subsubsection{i)}
            Die Projektion \(\left(\mathbb{D}^m\times\mathbb{D}^{n-m},\mathbb{S}^{m-1}\times\mathbb{D}^{n-m}\right)\to\left(\mathbb{D}^m,\mathbb{S}^{m-1}\right)\) ist eine Homotpie\"aquivalenz, und bildet einen Henkel auf seinen Kern ab. Da ein Henkel in \(\mathcal{W}_m\) stets an \(\mathcal{W}_{m-1}\) angebracht ist, gilt
            \begin{align*}
                H_k\left(\mathcal{W}_m,\mathcal{W}_{m-1}\right)&\cong\Tilde{H}_k\bigg(\bigvee_{j=1}^{c_m}\mathbb{D}^m\times\mathbb{D}^{n-m}\mathrel{/}\mathbb{S}^{m-1}\times\mathbb{D}^{n-m}\bigg)\\[-6pt]
                &\cong\bigoplus_{j=1}^{c_m}\Tilde{H}_k\left(\mathbb{D}^m\mathrel{/}\mathbb{S}^{m-1}\right)\cong\bigoplus_{j=1}^{c_m}\Tilde{H}_k\left(\mathbb{S}^m\right)\,.
            \end{align*}
            Demnach ist \(H_k\left(\mathcal{W}_k,\mathcal{W}_{k-1}\right)\) f\"ur \(m\not=k\) trivial und sonst zu \(\mathbb{Z}^{c_k}\) isomorph. Die Generatoren der \(\mathbb{Z}\)-Anteile stehen hierbei in bijektiver Korrespondenz zu den Kernen der Henkel, also auch zu den Henkeln selbst.
            
        \subsubsection{ii) \& iii)}
            Die lange exakte Folge des Tripels \(\left(\mathcal{W}_m,\mathcal{W}_{m-1},\mathcal{M}\right)\) ist gerade
            \begin{center}
                \begin{tikzpicture}
                    \path   node (A) at (-6, 1) {\(\dots\)}
                            node (B) at (-3.5, 1) {\(H_{k+1}\left(\mathcal{W}_m,\mathcal{W}_{m-1}\right)\)}
                            node (C) at (0, 1) {\(H_k\left(\mathcal{W}_{m-1},\mathcal{M}\right)\)}
                            node (D) at (0, 0) {\(H_k\left(\mathcal{W}_m,\mathcal{M}\right)\)}
                            node (E) at (3.1, 0) {\(H_k\left(\mathcal{W}_m,\mathcal{W}_{m-1}\right)\)}
                            node (F) at (5.4, 0) {\(\dots\)};
                    \draw[-stealth] (A.east) -- (B.west);
                    \draw[-stealth] (B.east) -- (C.west);
                    \draw[-stealth] (C.south) -- (D.north);
                    \draw[-stealth] (D.east) -- (E.west);
                    \draw[-stealth] (E.east) -- (F.west);
                \end{tikzpicture}
            \end{center}
            F\"ur \(m\not=k\) oder \(m\not=k-1\) ist entweder die erste oder letzte Gruppe trivial und die mittlere Abbildung damit ein Epimorphismus oder ein Monomorphismus. Dies zeigt ii). Mit \(H_k\left(\mathcal{W}_{-1},\mathcal{M}\right)=0\) ergibt dies iii).
    \end{proof} 
    Durch dem aus der langen exakten Folge stammenden Differential
    \begin{equation}\label{eq:differential}
        \partial_k\colon H_k\left(\mathcal{W}_k,\mathcal{W}_{k-1}\right)\to H_{k-1}\left(\mathcal{W}_{k-1},\mathcal{M}\right)
    \end{equation}
    und der von der Quotientenabbildung induzierten Abbildung 
    \[q_k^*\colon H_{k-1}\left(\mathcal{W}_{k-1},\mathcal{M}\right)\to H_{k-1}\left(\mathcal{W}_{k-1},\mathcal{W}_{k-2}\right)\]
    kann nun ein weiteres Differential durch \(\dx_k:=q_k^*\circ\partial_k\) definiert werden, welches im Folgenden genauer untersucht werde. Die Homologiegruppen \(H_k\left(\mathcal{W}_k,\mathcal{W}_{k-1}\right)\) mit diesem Differential als Zellkomplex ergeben neue Homologiegruppen \(H_k^{Hen}\left(\mathcal{W}\right)\). F\"ur diese gilt analog zu Hatcher Satz 2.35
    \[H_*^{Hen}\left(\mathcal{W}\right)\cong H_*\left(\mathcal{W},\mathcal{M}\right)\,,\]
    was jedoch im Folgenden nicht weiter ben\"otigt werde. 
    \begin{remark}\label{rem:incl_epi}
        Sei \(\left(\mathcal{W}^n,\mathcal{M},-\right)\) ein H-Kobordismus. Dann gilt wegen \(\mathcal{M}\simeq\mathcal{W}\) auch
        \[H_0\left(\mathcal{W}_1,\mathcal{M}\right)\cong H_0\left(\mathcal{W},\mathcal{M}\right)\cong0\,.\]
        Die lange exakte Folge des Paares \(\left(\mathcal{W}_1,\mathcal{M}\right)\) ist dann
        \[\dots\to H_1\left(\mathcal{W}_1,\mathcal{M}\right)\to H_0\left(\mathcal{M}\right)\mathop{\hookrightarrow}^{\iota^*}H_0\left(\mathcal{W}_1\right)\to0\to\dots\,,\]
        sodass \(\iota^*\) ein Epimorphismus ist. Folglich kann \(\mathcal{W}_1\) nicht mehr Zusammenhangskomponenten als \(\mathcal{M}\) besitzen.
    \end{remark}
    Existieren keine \(j\)-Henkel f\"ur \(j<k\), ist \(\mathcal{W}_{k-1}\cong\mathcal{M}\times\mathbb{I}\), und die Abbildung \(q_{k+1}^*\) ist gleich der Identit\"at, sodass \(\dx_{k+1}=\partial_{k+1}\) gilt. In diesem Fall existiert eine intuitive Beschreibung durch Schnittzahlen. 
    
    \subsection{Schnittzahlen und der Whitney-Trick}
        Seien \(\mathcal{M}^k\) und \(\mathcal{N}^{n-k}\) transversale Untermannigfaltigkeiten von \(\mathcal{W}^n\), wobei \(\mathcal{N}\) und das Normalenb\"undel von \(\mathcal{M}\) orientiert seien. Sei weiter \(\mathcal{M}\subset U\subseteq\mathcal{W}\) eine Tubenumgebung, sodass in allen Punkten \(p\in\mathcal{M}\cap\mathcal{N}\) die Faser \(U_p\) eine Umgebung von \(p\) in \(\mathcal{N}\) sei (aufgrund der Transversalit\"at kann dies stets gew\"ahrleistet werden). Dann ergeben sich induzierte lokale Orientierungen
        \[H_n\left(N\mathcal{M}\right)\cong H_n\left(U\right)\to H_n\left(U,U\setminus\mathcal{M}\right)\cong H_{n-k}\left(U_p,U_p\setminus 0\right)\]
        und
        \[H_{n-k}\left(\mathcal{N}\right)\cong H_{n-k}\left(\mathcal{N},\mathcal{N}\setminus p\right)\,.\]
        Da die Inklusion \(U_p\to\mathcal{N}\) Isomorphismen
        \[H_{n-k}\left(U_p,U_p\setminus p\right)\cong H_{n-k}\left(\mathcal{N},\mathcal{N}\setminus p\right)\]
        induziert, k\"onnen diese lokalen Orientierungen verglichen werden. In einem Punkt \(p\) sei nun \(\epsilon_p=1\), falls diese Orientierungen \"ubereinstimmen, und \(-1\) sonst. Die Schnittzahl sei
        \[\left[\mathcal{M},\mathcal{N}\right]:=\mathop{\sum\epsilon_p}_{p\hspace{1pt}\in\mathcal{M}\hspace{1pt}\cap\hspace{1pt}\mathcal{N}}\,.\]
        Seien nun alle Kerne von Henkeln \(\Psi^k,\Psi^{k+1}\) beliebig orientiert. Dies induziert sowohl eine Orientierung des Normalenb\"undels von \(\Sigma^k\), als auch eine Orientierung der Anklebesph\"are \(\Lambda^{k+1}\). Folglich k\"onnen im Falle eines transversalen Schnittes die Schnittzahlen \(\left[\Sigma^k,\Lambda^{k+1}\right]\) betrachtet werden, und es gilt (!)
        \[\dx\Psi^{k+1}=\sum_{j=1}^{c_k}\left[\Sigma_j^k,\Lambda^{k+1}\right]\Psi_j^k\,.\]
        
        \begin{proposition}[Whitneys Trick]
            Seien \(\mathcal{M}^m\) und \(\mathcal{N}^n\) zusammenh\"angende Untermannigfaltigkeiten der einfach zusammenh\"angenden Mannigfaltigkeit \(\mathcal{V}^{n+m}\), wobei \(n+m\geq5\), \(n\geq2\) und \(m\geq3\) gelte. Sei \(\mathcal{M}\) und das Normalenb\"undel von \(\mathcal{N}\) orientiert. Seien \(p,q\in\mathcal{M}\cap\mathcal{N}\) derart, dass \(\epsilon_p+\epsilon_q=0\) gelte, so existiert eine Isotopie \(h\colon\mathcal{M}\times\mathbb{I}\to\mathcal{V}\) die in einer Umgebung von \(\mathcal{M}\cap\mathcal{N}\setminus\{p,q\}\) station\"ar ist, und \(h_1\left(\mathcal{M}\right)\cap\mathcal{N}=\mathcal{M}\cap\mathcal{N}\setminus\{p,q\}\) gilt.
        \end{proposition}
        \begin{proof}
            Siehe \cite{milnor1965hcobordism} Seite 71 Satz 6.6.
            \renewcommand\qedsymbol{\(\cancel{qed}\)}
        \end{proof}
        Die Forderung der hohen Dimensionalit\"at ergibt sich daraus, dass eine \(2\)-dimensionale Scheibe in \(\mathcal{M}^n\) eingebettet werden muss, wof\"ur der schwache Einbettungssatz von Whitney ben\"otigt wird. Folglich muss \(n\geq 2\cdot2+1=5\) gelten. 
        
\section{Homologie- und Modifikationslemmata}
    Folgendes Lemma erkl\"art die simple Beziehung zwischen sich aufhebenden Henkeln und ist der Grund f\"ur die Forderung, dass der Kobordismus mindestens \(6\)-dimensional ist. Es ist eine (fast) direkte Folgerung aus dem Whitney-Trick.
    \begin{theorem}[Homologie-Lemma]\label{thm:homology}
        Besitzt eine Henkelzerlegung mit \(\mathcal{W}_{k-1}\cong\mathcal{M}\times\mathbb{I}\) eines mindestens \(6\)-di\-men\-sio\-na\-len Kobordismus \(\mathcal{W}\) keine \((j<k)\)-Henkel und gilt \(\dx\Psi^{k+1}=\Psi_i^k\), ist \(\Psi^{k+1}\) rechtsinvers zu \(\Psi_i^k\).
    \end{theorem}
    \begin{proof}
        Der Kern eines Henkels wird unter dem Differential auf seinen Rand, also auf die Anklebesph\"are abgebildet, die nach einer Isotopie als transversal zu allen G\"urteln von \(k\)-Henkeln angenommen werden kann. Aus Dimensiongr\"unden, sind dies lediglich endlich viele Punkte. Wegen
        \[\dx\Psi^{k+1}=\sum_{j=1}^{c_k}\Big[\Lambda^{k+1},\Sigma_j^k\Big]\Psi_j^k=\Psi_i^k\,,\]
        muss f\"ur \(j\not=i\) auch
        \[\Big[\Lambda^{k+1},\Sigma_j^k\Big]\mathop{=\sum\epsilon_p=}_{p\in\Sigma_j^k\cap\Lambda^{k+1}}0\]
        gelten, sodass stets Paare \(p,q\in\Lambda^{k+1}\cap\Sigma_j^k\) mit \(\epsilon_p+\epsilon_q=0\) existieren, die sich durch eine Isotopie gem\"a\ss{} dem Whitney-Trick aufheben. F\"ur \(i=j\) folgt analog, dass sich alle bis auf einen Punkt \(p\) mit \(\epsilon_p=1\) paarweise aufheben. Da sich \(\Psi^{k+1}\) und \(\Psi_i^{k+1}\) nun transversal und nur in \(p\) schneiden l\"asst sich der K\"urzungssatz anwenden.
    \end{proof}
    
    \begin{proposition}[Modifikationslemma]\label{prop:modification}
        F\"ur Henkel \(\Psi_1^k\) und \(\Psi_2^k\) existiert ein in \(\partial_+\mathcal{W}_k\) isotoper Henkel \(\Phi^k\simeq\Psi_1^k\) mit
        \[\dx\Phi^k=\dx\Psi_1^k+\dx\Psi_2^k\,.\]
    \end{proposition}
    \begin{proof}
        Dies folgt aus \cite{kosinski2013differential} Kapitel VIII Lemma 1.2.
    \end{proof}

    
    \newpage % -- Einfacher Zusammenhang von W_k zeigen --
    \chapter{Entfernung der 0- und 1-Henkel}\label{chp:remove_zero_one_handles}
        Um induktiv Henkel zu eliminieren, muss zun\"achst ein Induktionsanfang gezeigt werden. Hierzu m\"ussen die \(0\)- und \(1\)-Henkel eliminiert werden. 
\begin{theorem}[Entfernung der 0-Henkel]\label{thm:zero_handle_rem}
    Jeder H-Kobordismus besitzt eine Henkelzerlegung ohne \(0\)-Hen\-kel.
\end{theorem}
\begin{proof}
    Da die Anklebesph\"are eines \(0\)-Henkels \(\Psi^0\) leer ist, gilt trivialerweise
    \[\mathcal{M}\times\mathbb{I}+\Psi^0=\left(\mathcal{M}\times\mathbb{I}\right)\sqcup\mathbb{D}^n\,,\]
    sodass das Anbringen eines \(0\)-Henkels eine Wegzusammenhangskomponente hinzuf\"ugt. Aufgrund von Bemerkung \ref{rem:incl_epi} kann \(\mathcal{W}_1\) nicht mehr Wegzusammenhangskomponenten als \(\mathcal{M}\) besitzen. Demnach existiert ein \(1\)-Henkel \(\Psi^1\colon \mathbb{S}^0\to\mathcal{W}_0\), der \(\mathcal{M}\times\mathbb{I}\) mit \(\mathbb{D}^n\) verbindet. Die Anklebesph\"are von \(\Psi^1\) schneidet den G\"urtel von \(\Psi^0\), also die gesamte Sph\"are \(\mathbb{S}^{n-1}\), nun transversal und in genau einem Punkt. Folglich heben sich die Henkel gem\"a\ss{} Satz \ref{thm:handle_removement} auf.
\end{proof}
Um die \(1\)-Henkel zu entfernen ist erstmals vonn\"oten, dass der H-Ko\-bor\-dis\-mus einfach zusammen\-h\"angt.
\begin{theorem}[Entfernung der 1-Henkel]\label{thm:one_handle_rem}
    Jeder zusammenh\"angende, mindestens \(5\)-dimensionale H-Kobordismus besitzt eine Henkelzerlegung ohne \(0\)- oder \(1\)-Henkel.
\end{theorem}
\begin{proof}
    Zun\"achst sei \(\gamma_1\colon\mathbb{I}\to\mathbb{S}^{n-1}\setminus\mathbb{S}^0\subseteq\partial_+\mathcal{W}_1\) ein Weg mit Endpunkten \(x_1,x_2\in\partial_+\mathcal{W}_0\), der den G\"urtel von \(\Psi^1\) transversal und in genau einem Punkt schneide. Sind die \(0\)-Henkel gem\"a\ss{} Satz \ref{thm:zero_handle_rem} bereits eliminiert, ist \(\partial_+\mathcal{W}_0\cong\mathcal{M}\) wegzusammenh\"angend, und es existiert ein Weg \(\gamma_2\subseteq\partial_+\mathcal{W}_0\) von \(x_1\) zu \(x_2\), der keine Anklebesph\"are von \(1\)-Henkeln (endlich viele Punkte) treffe. Dann induziert \(\gamma_1\gamma_2\) eine stetige Abbildung \(\mathbb{S}^1\to\partial_+\mathcal{W}_1\), die mit dem schwachen Einbettungssatz von Whitney \ref{prop:whitney_weak_embedding} zuerst durch eine glatte Einbettung \(\Psi^1\) approximiert, und dann derart isotopiert werden kann, dass sie die Anklebesph\"aren aller \(2\)-Henkel transversal schneide. Die resultierende Einbettung sei durch \(\beta\) bezeichnet. Aus Dimensionsgr\"unden bedeute dies erneut, dass \(\beta\) von allen Anklebesph\"aren disjunkt ist und somit als Abbildung nach \(\partial_+\mathcal{W}_2\) aufgefasst werden kann. Nun existiert stets ein trivialer Henkel \(\Psi^2\), dessen Anklebeabbildung wegen \(\pi_1\left(\partial_+\mathcal{W}_2\right)=0\) (!) homotop zu \(\beta\) ist. Wegen \(1\cdot2+2=4\leq\dim\partial_+\mathcal{W}\), sind diese gem\"a\ss{} Satz \ref{prop:whitney_unique_embedding} in \(\partial_+\mathcal{W}_2\) sogar zueinander isotop, und Korollar \ref{cor:handle_replacement} erm\"oglicht das Ersetzen von \(\Psi^1\).
\end{proof}
        
    \newpage % -- Fertig --
    \chapter{Der H-Kobordismus-Satz}\label{chp:h_cobordism_theorem} 
        \section{Die Normalform}
    \begin{theorem}[Existenz einer Normalform]\label{thm:normal_form}
        Jeder einfach zusammenh\"angende \((n\geq6)\)-dimensionalen H-Kobordismus besitzt f\"ur jedes \(2\leq k\leq n-3\) eine Henkelzerlegung, die lediglich \(k\)- und \((k+1)\)-Henkel besitzt. 
    \end{theorem}
    \begin{proof}
        Es wird \"uber Induktion nach \(k\) gezeigt, dass eine Darstellung ohne \(j\)-Henkel f\"ur \(j<k\) existiert, also dass \(\mathcal{W}_{k-1}\cong\mathcal{M}\times\mathbb{I}\) gilt. Der Induktionsanfang \(k=2\) ist gerade die Aussage von Satz \ref{thm:one_handle_rem}.
            
        \subsubsection*{Induktionsschritt}
            Sei die Aussage f\"ur \(k\) bereits gezeigt, dann existiert eine Henkelzerlegung von \(\mathcal{W}\) mit \(\mathcal{W}_{k-1}=\mathcal{M}\times\mathbb{I}\) und f\"ur das Differential in den Homologiegruppen gilt \(\dx_{k+1}=\partial_{k+1}\). Dieses ist wegen der langen exakten Folge
            \begin{center}
                \begin{tikzpicture}
                    \path   node (A) at (-6, 1) {\(\dots\)}
                            node (B) at (-3.5, 1) {\(H_{k+1}\left(\mathcal{W}_{k+1},\mathcal{W}_k\right)\)}
                            node (C) at (0, 1) {\(H_k\left(\mathcal{W}_k,\mathcal{M}\right)\)}
                            node (D) at (0, 0) {\(H_k\left(\mathcal{W}_{k+1},\mathcal{M}\right)\cong H_k\left(\mathcal{W},\mathcal{M}\right)\cong0\)}
                            node (E) at (3.7, 0) {\(\dots\)};
                    \draw[-stealth] (A.east) -- (B.west);
                    \draw[-stealth] (B.east) -- node[above, pos = 0.5] {\scriptsize\(\partial_{k+1}\)} (C.west);
                    \draw[-stealth] (C.south) -- (D.north);
                    \draw[-stealth] (D.east) -- (E.west);
                \end{tikzpicture}
            \end{center}
            ein Epimorphismus. Somit existiert f\"ur alle \(\Psi^k\in H_k\left(\mathcal{W}_k,\mathcal{M}\right)\) ein Element 
            \[\sum_{j=1}^{c_{k+1}}x_j\Psi_j^{k+1}\in H_{k+1}\left(\mathcal{W}_{k+1},\mathcal{W}_k\right),\,\quad\text{sodass}\quad\sum_{j=1}^{c_{k+1}}x_j\dx\Psi_j^{k+1}=\Psi^k\quad\text{gilt.}\]
            Sei \(\Phi^{k+1}\) ein trivialer Henkel, ist die Anklebesph\"are kontrahierbar, und somit \(\dx\Phi^{k+1}=0\). Folglich existiert gem\"a\ss{} dem Modifikationssatz \ref{prop:modification} ein zu \(\Phi^{k+1}\) isotoper Henkel \(\chi^{k+1}\) mit
            \[\dx\chi^{k+1}=\dx\Phi^{k+1}+\sum_{j=1}^{c_{k+1}}x_j\dx\Psi_j^{k+1}=0+\Psi^k=\Psi^k\,.\]
            Gem\"a\ss{} dem Homologie-Lemma \ref{thm:homology} existiert ein zu \(\chi^{k+1}\) isotoper Henkel, der den G\"urtel von \(\Psi^k\) transversal und in genau einem Punkt schneide, sodass Korollar \ref{cor:handle_replacement} die Ersetzung von \(\Psi^k\) mit einem \((k+2)\)-Henkel erm\"oglicht. 
            
            Durch \"Ubergang zu der dualen Repr\"asentation k\"onnen \(j\)-Henkel f\"ur \(j>k+1\) in \((n-j)\)-Henkel \"uberf\"uhrt werden und eine Anwendung der oberen Induktion mit \(k^{\prime}=n-k\) zeigt die Aussage.
    \end{proof}
    
\section{Differentialmatrizen}
    Sei \(H^k\left(\mathcal{W}_k,\mathcal{W}_{k-1}\right)\cong\mathbb{Z}^{c_k}\) als \(\mathbb{Z}\)-Modul mit der kanonischen Basis aufgefasst, kann die Abbildungsmatrix \(M_k\in\mathbb{Z}^{c_k\times c_{k-1}}\) des Differentials \(\dx_k\) definiert werden. Diese ist stets von der gegebenen Henkelzerlegung abh\"angig. Aus der Existenz einer Normalform folgt, dass lediglich ein \(k\) betrachtet werden muss. Elementare Zeilenoperationen auf dieser Matrix ergeben nun Matrizen, die mit weiteren Henkelzerlegungen von \(\mathcal{W}\) korrespondieren. Somit kann ein H-Kobordismus und das Entfernen oder Hinzuf\"ugen von Henkeln durch wenige Daten beschrieben, und mithilfe des Gau\ss schen Eliminationsverfahrens deutlich vereinfacht werden. 
    
    \subsection{Elementare Zeilenoperationen}
        Dass eine Zeile zu einer andere Zeile addiert werden kann folgt aus dem Modifikationslemma. Ist die Matrix von der Form 
        \[A=\begin{pmatrix}
            B & 0\\
            0 & 1
        \end{pmatrix}\,,\]
        ergibt das Homologie-Lemma \ref{thm:homology} eine Henkelzerlegung mit Differentialmatrix \(B\). Zeilenvertauschungen korrespondieren mit einer Umnummerierung der Henkel und eine Skalierung einer Zeile mit \(-1\) ist \"aquivalent mit der Umorientierung des zugeh\"origen Kerns.
        
    \begin{theorem}[H-Kobordismus-Satz]
        Jeder mindestens \(6\)-dimensionale, einfach zusammenh\"angende H-Ko\-bor\-dis\-mus \((\mathcal{W},\mathcal{M},-)\) ist trivial.
    \end{theorem}
    \begin{proof}
        Anwenden des gau\ss schen Eliminationsverfahrens auf die Differentialmatrix einer Normalform ergibt eine Henkelzerlegung, deren Differentialmatrix die Identit\"at ist. Wiederholte Reduktion dieser resultiert in der leeren Matrix, die eine Henkelzerlegung ohne Henkel repr\"asentiert. Es folgt 
        \[\mathcal{W}\cong\mathcal{W}_n\cong\mathcal{W}_{-1}=\mathcal{M}\times\mathbb{I}\,.\]
    \end{proof}

    \newpage
    \refstepcounter{chapter}
    \addcontentsline{toc}{chapter}{Appendix}
    \chapter*{Appendix}
        \subsection*{Wohldefiniertheit des Diffeomorphismus aus Satz \ref{lem:glueing_disc_bundles}}\label{app:diff_well_defined}
    Sei
    \[\Phi(z):=\begin{cases}
        z & z\in\mathcal{M}\setminus\mathcal{V}\\
        \Tilde{h}_2\left(\frac{2v+\left(1-\norm{v^2}-t^2\right)\pi(z)}{\norm{v^2}+(1-t)^2}\right) & z=tp+v\in\mathcal{B}\setminus\mathcal{V}^{\prime}
    \end{cases}\,,\]
    und \(z=v+tp\in\left(\mathcal{V}^{\prime}\right)^{\perp}\setminus\mathbf{0}\) die orthogonale Zerlegung bez\"uglich der riemannschen Metrik. Dann gilt
    \begin{align}\label{eq:diff_well_defined_identification}
        \begin{split}
            \Tilde{h}_2(z)\sim\Tilde{h}_1\circ\alpha(z)&=\frac{\frac{2v}{\norm{z}^2}+\left(1-\norm{\frac{v}{\norm{z}^2}}^2-\left(\frac{t}{\norm{z}^2}\right)^2\right)p}{\norm{\frac{v}{\norm{z}^2}}^2+\left(1+\frac{t}{\norm{z}^2}\right)^2}\\
            &=\frac{\frac{2v}{\norm{z}^2}+\left(1-\frac{\norm{v}^2}{\norm{z}^4}-\frac{t^2}{\norm{z}^4}\right)p}{\frac{\norm{v}^2}{\norm{z}^4}+\left(1+\frac{t}{\norm{z}^2}\right)^2}\\
            &=\frac{\frac{2v}{\norm{z}^2}+\left(1-\frac{\norm{z}^2}{\norm{z}^4}\right)p}{\frac{\norm{v}^2}{\norm{z}^4}+1+\frac{2t}{\norm{z}^2}+\frac{t^2}{\norm{z}^4}}\\
            &=\frac{\frac{1}{\norm{z}^2}\left(2v+\left(\norm{v}^2+t^2-1\right)p\right)}{\frac{1}{\norm{z}^2}\left(1+\norm{z}^2+2t\right)}\\
            &=\frac{2v+\left(\norm{v}^2+t^2-1\right)p}{\norm{v}^2+\left(1+t\right)^2}\,.
        \end{split}
     \end{align}
     Sei nun
    \[\beta:=\norm{v}^2+\left(1+t\right)^2\,,\]
    so gilt weiter
    \begin{align}\label{eq:diff_well_defined_result}
        \begin{split}
            \Tilde{h}_2 ^{-1}\circ\Phi\left(\Tilde{h}_1\circ\alpha(z)\right)&\mathop{=}^{\eqref{eq:diff_well_defined_identification}}\Phi\left(\frac{2v}{\beta}+\frac{\norm{v}^2+t^2-1}{\beta}\,p\right)\\
            &=\frac{\frac{4v}{\beta}+\left(1-\frac{4\norm{v}^2}{\beta^2}-\left(\frac{\norm{v}^2+t^2-1}{\beta}\right)^2\right)p}{\frac{4\norm{v}}{\beta^2}+\left(1-\frac{\norm{v}^2+t^2-1}{\beta}\right)^2}\\
            &=\frac{\frac{4v}{\beta}+\frac{\beta^2-4\norm{v^2}-(\norm{v}^2+t^2-1)^2}{\beta^2}\,p}{\frac{4\norm{v}}{\beta^2}+\left(1-\frac{\norm{v}^2+t^2-1}{\beta}\right)^2}\\
            &\mathop{=}^{\eqref{eq:diff_well_defined_calc_1}}\frac{\frac{4v}{\beta}+\frac{4t\beta p}{\beta^2}}{\frac{4\norm{v}}{\beta^2}+\left(1-\frac{\norm{v}^2+t^2-1}{\beta}\right)^2}\\
            &\mathop{=}^{\eqref{eq:diff_well_defined_calc_2}}\frac{\frac{4v}{\beta}+\frac{4t\beta}{\beta^2}\,p}{\frac{4\norm{v}}{\beta^2}+\frac{4(1+t)^2}{\beta^2}}=\frac{\frac{4}{\beta}\left(v+tp\right)}{\frac{4}{\beta}\left(\frac{\norm{v}^2+(1+t)^2}{\beta}\right)}\\
            &=v+tp=z\,,
        \end{split}
    \end{align}
    wobei
    \begin{align}\label{eq:diff_well_defined_calc_1}
        \begin{split}
            &\beta^2-4\norm{v^2}-(\norm{v}^2+t^2-1)^2\\
            &=\norm{v}^4+2\norm{v}^2(1+t)^2+(1+t)^4-4\norm{v}^2\\
            &-\norm{v}^4-2\norm{v}^2(t^2-1)-(t^2-1)^2\\
            &=\norm{v}^4+2\norm{v}^2+4\norm{v}^2t+2\norm{v}^2t^2+1+4t+6t^2+4t^3+t^4-4\norm{v}^2\\
            &-\norm{v}^4-2\norm{v}^2t^2+2\norm{v}^2-t^4+2t^2-1\\
            &=4\norm{v}^2t+4t+8t^2+4t^3\\
            &=4t\left(\norm{v}^2+1+2t+t^2\right)=4t\beta
        \end{split}
    \end{align}
    und
    \begin{align}\label{eq:diff_well_defined_calc_2}
        \begin{split}
            \left(1-\frac{\norm{v}^2+t^2-1}{\beta}\right)^2&=\frac{\left(\norm{v}^2+1+2t+t^2-\norm{v}^2-t^2+1\right)^2}{\beta^2}\\
            &=\frac{4\left(1+t\right)^2}{\beta^2}
        \end{split}
    \end{align}
    seien. Schlie\ss lich ist also wegen \(\Tilde{h}_2(z)\in\mathcal{M}\setminus\mathcal{V}\)
    \[\Phi\left(\Tilde{h}_2(z)\right)\mathop{=}^{\text{Def.}}\Tilde{h}_2(z)\mathop{=}^{\eqref{eq:diff_well_defined_result}}\Phi\left(\Tilde{h}_1\circ\alpha(z)\right)\,,\]
    was die Aussage zeigt. \qed

\newpage
\section*{Tubenumgebung der Sph\"are}\label{app:sphere_tub_emb}
    \subsection*{Vor\"uberlegungen}
        Zun\"achst seien die Diffeomorphismen
        \[f\colon\mathbb{R}_{>0}\to\,]0,1[,\,x\mapsto\frac{1}{\sqrt{1+x^2}}\quad\text{und}\quad g\colon]0,1[\,\to\mathbb{R}_{>0},\,x\mapsto\frac{\sqrt{1-x^2}}{x}\]
        definiert. Diese sind zueinander invers, und es gilt
        \begin{equation}\label{eq:app_tub_01}
            f\left(\frac{1}{g(x)}\right)=\sqrt{1-x^2}\,.
        \end{equation}
        Weiter existieren die Diffeomorphismen
        \[\Phi\colon\mathring{\mathbb{D}}^k\times\mathbb{D}^{n-k}\to\mathbb{D}^n\setminus\mathbb{S}^{k-1},\,(x,y)\mapsto\left(x,y\sqrt{1-\norm{x}^2}\right)\]
        und f\"ur \(\mathring{H}^k:=\left(\mathring{\mathbb{D}}^k\setminus\{0\}\right)\times\mathbb{D}^{n-k}\) auch
        \[\Lambda\colon\mathring{H}^k\to\mathring{H}^k,\,(x,y)\mapsto\left(x\frac{\sqrt{1-\norm{x}^2}}{\norm{x}},y\right)\,,\]
        wobei \(\Lambda\) gleichzeitig eine Involution ist.

    \subsection*{Die Tubenumgebung}
        Als Vektorb\"undel sei 
        \[\pi\colon E\to\mathbb{S}^{k-1},\,(p,v)\mapsto p\quad\text{mit}\quad E:=\mathbb{S}^{k-1}\times\mathbb{R}^{n-k}\]
        gew\"ahlt. Betrachte den Diffeomorphismus
        \[G\colon E\times\mathbb{R}_{\geq0}\setminus\mathbf{0}\to\mathring{H}^k,\,(p,z)\mapsto\left(f(\norm{z})\cdot p,\frac{x}{\norm{z}}\right)\]
        mit der Inversen
        \[G^{-1}\colon\mathring{H}^k\to E\times\mathbb{R}_{\geq0}\setminus\mathbf{0},\,(x,y)\mapsto\left(\frac{x}{\norm{x}},g\left(\norm{x}\right)\left(y,\sqrt{1-\norm{y}^2}\right)\right)\,.\]
        Die gesuchte Einbettung ist nun 
        \[\Tilde{h}\colon E\times\mathbb{R}_{\geq0}\to U\subset\mathbb{D}^n,\,(p,z)\mapsto\begin{cases}
            F\circ G(p,z) & z\not=0\\
            (p,0) & \text{sonst}
        \end{cases}\,.\]
    
    \subsection*{Alpha}
        Es verbleibt zu zeigen, dass bei dieser Wahl der Einbettung \(\Tilde{h}\circ\alpha_E\circ\Tilde{h}^{-1}=\alpha\) gilt. Zun\"achst sei angemerkt, dass \(\alpha=F^{-1}\circ\Lambda\circ F\) ist, sodass lediglich \(\Lambda=G\circ\alpha_E\circ G^{-1}\) gezeigt werden muss. Die Aussage folgt aus
        \begin{align*}
            G\circ\alpha_E\circ G^{-1}(x,y)&=G\circ\alpha_E\left(\frac{x}{\norm{x}},g\left(\norm{x}\right)\cdot y,g\left(\norm{x}\right)\sqrt{1-\norm{y}^2}\right)\\
            &=G\left(\frac{x}{\norm{x}},\frac{y}{g(\norm{z})},\frac{\sqrt{1-\norm{y}^2}}{g(\norm{z})}\right)\\
            &=\left(f\left(\frac{1}{g(\norm{z})}\right)\frac{x}{\norm{x}},\frac{y}{g(\norm{z})}g(\norm{z})\right)\\
            &\mathop{=}^{\eqref{eq:app_tub_01}}\left(x\frac{\sqrt{1-\norm{x}^2}}{\norm{x}},y\right)=\Lambda(x,y)\,.
        \end{align*}
    
\newpage
\subsection*{Wohldefiniertheit der Projektion aus Satz \ref{lem:handle_on_unit_disc}}\label{app:disc_bundle_well_defined}
    Zun\"achst gilt f\"ur \(z=(x,y)\in U\)
    \begin{align}\label{eq:disc_bundle_well_defined_calc_1}
        \begin{split}
            \Tilde{\Psi}(z)&=\norm{z}\Psi\left(\frac{z}{\norm{z}}\right)\\
            &=\norm{z}\left(\norm{\frac{x}{\norm{z}}}\psi\left(\frac{\frac{x}{\norm{z}}}{\frac{x}{\norm{z}}}\right),\gamma\left(\frac{\frac{x}{\norm{z}}}{\frac{x}{\norm{z}}}\right)\frac{y}{\norm{z}}\right)\\
            &=\left(\norm{x}\psi\left(\frac{x}{\norm{x}}\right),\gamma\left(\frac{x}{\norm{x}}\right)y\right)
        \end{split}
    \end{align}
    also
    \begin{align*}
        \Tilde{\Psi}\circ\alpha(z)&=\Tilde{\Psi}\left(x\frac{\sqrt{1-\norm{x}^2}}{\norm{x}},y\frac{\norm{x}}{\sqrt{1-\norm{x}^2}}\right)\\
        &\mathop{=}^{\eqref{eq:disc_bundle_well_defined_calc_1}}\left(\sqrt{1-\norm{x}^2}\psi\left(\frac{x}{\norm{x}}\right),\frac{\norm{x}}{\sqrt{1-\norm{x}^2}}\gamma\left(\frac{x}{\norm{x}}\right)y\right)
    \end{align*}
    und somit 
    \[\pi\left(\Tilde{\Psi}\circ\alpha(z)\right)=\sqrt{1-\norm{x}^2}\psi\left(\frac{x}{\norm{x}}\right)\,,\]
    sowie
    \[\Tilde{\Psi}\circ\alpha(\pi(z))=\left(\sqrt{1-\norm{x}^2}\psi\left(\frac{x}{\norm{x}}\right),0\right)\,.\]

        
    \addcontentsline{toc}{chapter}{Literaturverzeichnis}
    \bibliographystyle{alpha}
    \bibliography{sources}
\end{document}
