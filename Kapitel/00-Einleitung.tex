Der H-Kobordismus-Satz ist ein wichtiges Resultat der Differentialtopologie und erm\"oglicht unter anderem, in hinreichend gro\ss er Dimension zu entscheiden, ob zwei Mannigfaltigkeiten zueinander diffeomorph sind. Mit diesem l\"asst sich unter anderem die verallgemeinerte Poincar\'e-Vermutung f\"ur Dimensionen \(n\geq5\) beweisen, wof\"ur Stephen Smale 1966 die Fields-Medaille erhielt. Weitere Beweise des Satzes lassen sich zum Beispiel in den B\"uchern \glqq Differential Manifolds\grqq{} von Antoni Kosinski \cite{kosinski2013differential}, \glqq Lectures on the h-Cobordism Theorem\grqq{}, welches auf den Niederschriften von L. Siebenmann und J. Sondow eines Seminares von John Milnor basiert \cite{milnor1965hcobordism}, oder \glqq A Basic Introduction to Surgery Theory\grqq{} von Wolfgang L\"uck \cite{lück2002surgery} finden. Der Ansatz von Milnor ist hierbei einer sehr technischen Natur kommt komplett ohne die Erw\"ahnung von Henkelzerlegungen aus. Ebenso vollst\"andig und technisch ist der Ansatz von Kosinski. Da der H-Kobordismus Satz in L\"ucks Buch nur einen kurzen Teil einnimmt, ist der dort dargelegte Beweis etwas k\"urzer und nicht auf Details fokussiert. Ziel der folgenden Arbeit sei, die Beweiskette von L\"uck zu verfolgen, wobei einige Details und Definitionen eher an Kosinski angelehnt seien. Grundlagen \"uber glatte Mannigfaltigkeiten lassen sich zum Beispiel in \glqq Einf\"uhrung in die Differentialtopologie\grqq{} von Klaus J\"anich und Theodor Br\"ocker \cite{bröcker1990difftop}, \glqq Differential Topology\grqq{} von Morris Hirsch \cite{hirsch2012difftop} oder \glqq Introduction to Smooth Manifolds\grqq{} von John M. Lee \cite{lee2013introduction}. Jegliche Ergebnisse der (algebraischen) Topologie k\"onnen in \glqq Algebraic Topology\grqq{} von Allen Hatcher \cite{hatcher2002algebraic} gefunden werden. Letztlich sei noch angemerkt, dass der Inhalt des Appendix zwar f\"ur die eigentliche Arbeit ignoriert werden kann, jedoch einen enormen Zeitaufwand erfordert hat.